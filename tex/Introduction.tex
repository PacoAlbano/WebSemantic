\section{Introduction}

The Semantic Web extends the Web to make data between computers easier to exchange and easier to use. The term "Granada"
can be supplemented, for example, in a web document with the information as to whether a ship name, family name or town
name is meant here. This additional information explicates the otherwise unstructured data. Standards are used for the
publication and use of machine-readable data. While people can close such information from the given context and
unconsciously build such links, machines must first be brought to this context. For this purpose, the content is linked
with further information and Web Ontology Language (OWL) is a way to write domains and their relationships formally. OWL
is a specification of the World Wide Web Consortium (W3C) to create, publish and distribute ontologies using a formal
description language.\\

This paper therefore creates a an power grid ontology as a general use case in Section II. Section III delineates the general challenge arising with the regard to the integration of Java Application to OWL Ontologies and presents existing technical approaches. Section IV provides an outlook and discusses the suitability for use cases with focus on integration and multi-platform approaches.
\newpage